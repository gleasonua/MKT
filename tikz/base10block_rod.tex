\documentclass{standalone}

\usepackage{amssymb}
\usepackage{amsthm}
\usepackage{amsmath}


\usepackage{tikz}
\usetikzlibrary{shapes,backgrounds,calc,patterns}
\usepackage{venndiagram}


\begin{document}
    \usetikzlibrary{calc,intersections}
\begin{tikzpicture}[scale=.5]
\node at (0,2) {\large Rod};
\clip (-1,-3) rectangle (1,1);
\coordinate (tf) at (0,0);
\coordinate (bf) at (0,-3);
\coordinate (tr) at (15:.25cm);
\coordinate (tl) at (165:.25cm);

% You can change the perspective by playing with the 5, 5, 15:
\coordinate (fr) at ($ (tf)!20!(tr) $);
\coordinate (fl) at ($ (tf)!15!(tl) $);
\coordinate (fb) at ($ (tf)!15!(bf) $);

\path[name path=brpath] (bf) -- (fr);
\path[name path=rbpath] (tr) -- (fb);
\path[name path=blpath] (bf) -- (fl);
\path[name path=lbpath] (tl) -- (fb);
\path[name path=trpath] (tl) -- (fr);
\path[name path=tlpath] (tr) -- (fl);

\draw[name intersections={of=brpath and rbpath}] (intersection-1)coordinate (br){}; 
\draw[name intersections={of=blpath and lbpath}] (intersection-1)coordinate (bl){}; 
\draw[name intersections={of=trpath and tlpath}] (intersection-1)coordinate (tb){}; 

\shade[right color=gray!10, left color=black!50, shading angle=105] (tf) -- (bf) -- (bl) -- (tl) -- cycle;
\shade[left color=gray!10, right color=black!50, shading angle=75] (tf) -- (bf) -- (br) -- (tr) -- cycle;

\begin{scope}
\clip (tf) -- (tr) -- (tb) -- (tl) -- cycle;
\shade[inner color = gray!5, outer color=black!50, shading=radial] (tf) ellipse (3cm and 1.5cm);
\end{scope}

\draw (tf) -- (bf);
\draw (tf) -- (tr);
\draw (tf) -- (tl);
\draw (tr) -- (br);
\draw (bf) -- (br);
\draw (tl) -- (bl);
\draw (bf) -- (bl);
\draw (tb) -- (tr);
\draw (tb) -- (tl);

%set the sizes of the little cubes:
\def\tone{.1}\def\ttwo{.2}\def\tthree{.3}\def\tfour{.4}\def\tfive{.5}\def\tsix{.6}\def\tseven{.7}\def\teight{.8}\def\tnine{.9}
\def\fone{.1}\def\ftwo{.2}\def\fthree{.3}\def\ffour{.4}\def\ffive{.5}\def\fsix{.6}\def\fseven{.7}\def\feight{.8}\def\fnine{.9}


\draw ($ (tl)!\fone!(bl) $) -- ($ (tf)!\fone!(bf) $) -- ($ (tr)!\fone!(br) $);
\draw ($ (tl)!\ftwo!(bl) $) -- ($ (tf)!\ftwo!(bf) $) -- ($ (tr)!\ftwo!(br) $);
\draw ($ (tl)!\fthree!(bl) $) -- ($ (tf)!\fthree!(bf) $) -- ($ (tr)!\fthree!(br) $);
\draw ($ (tl)!\ffour!(bl) $) -- ($ (tf)!\ffour!(bf) $) -- ($ (tr)!\ffour!(br) $);
\draw ($ (tl)!\ffive!(bl) $) -- ($ (tf)!\ffive!(bf) $) -- ($ (tr)!\ffive!(br) $);
\draw ($ (tl)!\fsix!(bl) $) -- ($ (tf)!\fsix!(bf) $) -- ($ (tr)!\fsix!(br) $);
\draw ($ (tl)!\fseven!(bl) $) -- ($ (tf)!\fseven!(bf) $) -- ($ (tr)!\fseven!(br) $);
\draw ($ (tl)!\feight!(bl) $) -- ($ (tf)!\feight!(bf) $) -- ($ (tr)!\feight!(br) $);
\draw ($ (tl)!\fnine!(bl) $) -- ($ (tf)!\fnine!(bf) $) -- ($ (tr)!\fnine!(br) $);   
\end{tikzpicture}

\end{document}